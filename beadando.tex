\documentclass[11pt,twoside,a4pape,draftr]{article}
	\usepackage[iso]{datetime}
	\usepackage[hungarian]{babel}

	\title{Nemlineáris egyenletrendszerek megoldása}
	\date{\today}
	\author{Filep Illés Attila}

\begin{document}
 	\maketitle
	\pagenumbering{gobble}
  	\newpage
  	\pagenumbering{arabic}

	\begin{abstract}

	\end{abstract}
	\tableofcontents
	\section*{Bevezetés}
	\section{Nemlineáris egyenletrendszerek bemutatása}
	\section{Nemlineáris egyenletrendszerek megoldására szolgáló módszerek bemutatása}
		\paragraph{}
			A fejezetben a bemutatni kívánt módszerek felületes leírása található.
		\subsection{fokozatos közelítés módszere}
			\paragraph{}
				Ismerhetjük még, mint fixpont iterációs módszer vagy egyszerű iterációs módszer.
		\subsection{általánosított Newton módszer}
			\paragraph{}
				Ez a módszert ebben a formában ritkán használjuk, mert különböző erre épülő módszerek vették át a helyét, a teljesség igénye nélkül ilyenek a: Quasi-Newton-módszer,  Variable Metric-módszer,  DFP (Davidon-Fletcher-Powell) algoritmus, 	BFGS (Broyden-Fletcher-Goldfarb-Shanno) algoritmus.
		\subsection{Broyden módszere}
			\paragraph{}
				A módszert még ezeken a neveken ismerhetjük: Broyden-féle secant módszer,  Broyden-féle quasi-Newton-módszer,  Broyden's good method
	\section{Számpélda(ák)}
		\paragraph{}
			Az [kiválasztott] módszernek három iterációját fogom bemutatni a következő számpéldán:
	\section{Tétel kimondása és bizonyítása}
	\section{Demonstráció}
		\subsection{feladat bemutatása}
		\subsection{Megoldás bemutatása}
\end{document}